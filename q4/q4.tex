\documentclass{article}
\usepackage{fullpage,amsmath,amssymb}
\usepackage{hyperref}
\usepackage[none]{hyphenat}
\usepackage{calc}  % arithmetic in length parameters
\usepackage{caption}
\usepackage{enumitem}  % more control over list formatting
\usepackage{fancyhdr}  % simpler headers and footers
\usepackage{geometry}  % page layout
\usepackage{lastpage}  % for last page number
\usepackage{listings}
\usepackage{relsize}  % easier font size changes
\usepackage[normalem]{ulem}  % smarter underlining
\usepackage{url}  % verb-like typesetting of URLs
\usepackage{xcolor}
\usepackage{xfrac}  % nicer looking simple fractions for text and math

\definecolor{codegray}{rgb}{0.5,0.5,0.5}
\definecolor{codegreen}{rgb}{0,0.6,0}
\definecolor{codepurple}{rgb}{0.58,0,0.82}
\definecolor{backcolour}{rgb}{0.95,0.95,0.92}

\lstdefinestyle{mystyle}{
    backgroundcolor=\color{backcolour},   
    commentstyle=\color{codegreen},
    keywordstyle=\color{magenta},
    numberstyle=\tiny\color{codegray},
    stringstyle=\color{codepurple},
    basicstyle=\ttfamily\footnotesize,
    breakatwhitespace=false,         
    breaklines=true,                 
    captionpos=b,                    
    keepspaces=false,                 
    numbers=left,                    
    numbersep=9pt,                  
    showspaces=false,                
    showstringspaces=false,
    showtabs=false,                  
    tabsize=4
}
\lstset{style=mystyle}

\everymath{\displaystyle}

\newcommand{\arrayex}[1]{
    \begin{tabular}{|*{20}{c|}}
    \hline
    #1 \\
    \hline
    \end{tabular}
}

\setlength{\tabcolsep}{9pt}

\renewcommand{\arraystretch}{1}

%\usepackage[T1]{fontenc}  % use true 8-bit fonts
%\usepackage{slantsc}  % allow slanted small-caps
%\usepackage{microtype}  % perform various font optimizations
%% Use Palatino-based monospace instead of kpfonts' default.
%\usepackage{newpxtext}

% Common macros.
\input{macros-263}


\geometry{a4paper, margin=1in, headheight=15pt, headsep=20pt}

\pagestyle{fancy}
\fancyhf{}
\fancyhead[L]{CSC311 Summer 2024}
\fancyhead[R]{Final Project}
\fancyfoot[C]{\thepage}


\title{Question 4}
\date{\vspace{-7.5ex}}
\hypersetup{pdfpagemode=Fullscreen,
    colorlinks=true,
    linkfileprefix={}}


\begin{document}
\maketitle
\thispagestyle{fancy}


\begin{enumerate}
    The final validation accuracy is: 0.66286 \\
    The final test accuracy is: 0.66949 \vspace{10} \\
    \textbf{Ensemble process:} \\
    we use three neural network models to implemented bagging ensemble. We first randomly sample three sample with replacement from out training data set. Then we train three different neural network independently for each training sample. These three neural network are complete independent and can run individually. After all models are trained, we use them to make prediction separately, finally we take the average of each of their predictions as our final prediction. \vspace{10}\\
    \textbf{Better or Not:} \\
    No, the bagging model is nearly the same performance as the single neural network model, so it doesn't improve the performance.\vspace{10} \\
    \textbf{Reason:}\\
    Ensembling the same model which train on different data subset has lack model diversity, thus it does not always improve the model performance. Also small training subset could be another problem, when the training set is small, there could be a issue that training subset are even smaller so that each model are not well trained, which result in poor performance.
    

\end{enumerate}

\end{document}
