%%  Common macros for the course CSC263H1 at the University of Toronto.
%%
%%  Copyright (c) 2014 Francois Pitt <fpitt@cs.utoronto.ca>
%%  last updated at 17:53 (EDT) on Sun 19 Oct 2014
%%
%%  CC BY-SA 4.0
%%  This work (the current file named 'macros-263.tex') is licensed under
%%  the Creative Commons Attribution-ShareAlike 4.0 International License.
%%  To view a copy of this license, visit
%%      http://creativecommons.org/licenses/by-sa/4.0/
%%  or send a letter to: Creative Commons, 444 Castro Street, Suite 900,
%%  Mountain View, California, 94041, USA.
%%  This is a human-readable summary of (and not a substitute for) the
%%  license.
%%  You are free to:
%%      Share -- copy and redistribute the material in any medium or format
%%      Adapt -- remix, transform, and build upon the material for any
%%          purpose, even commercially.
%%      The licensor cannot revoke these freedoms as long as you follow the
%%          license terms.
%%  Under the following terms:
%%      Attribution -- You must give appropriate credit, provide a link to
%%          the license, and indicate if changes were made. You may do so in
%%          any reasonable manner, but not in any way that suggests the
%%          licensor endorses you or your use.
%%      ShareAlike -- If you remix, transform, or build upon the material,
%%          you must distribute your contributions under the same license as
%%          the original.
%%      No additional restrictions -- You may not apply legal terms or
%%          technological measures that legally restrict others from doing
%%          anything the license permits.
%%  Notices:
%%      You do not have to comply with the license for elements of the
%%      material in the public domain or where your use is permitted by an
%%      applicable exception or limitation.
%%      No warranties are given. The license may not give you all of the
%%      permissions necessary for your intended use. For example, other
%%      rights such as publicity, privacy, or moral rights may limit how you
%%      use the material.

% Redefine \today in the style "DD Month YYYY".
\renewcommand*\today
 {\number\day\space\ifcase\month\or January\or February\or March\or
  April\or May\or June\or July\or August\or September\or October\or
  November\or December\fi\space\number\year}

% TeX trick so that math symbols are bold when text is.
\let\seiresfb\bfseries\def\bfseries{\boldmath\seiresfb}
\let\seiresdm\mdseries\def\mdseries{\unboldmath\seiresdm}

% Centered version of \llap and \rlap.
\providecommand*\clap[1]{\hbox to 0pt{\hss#1\hss}}

% Spacing macros with default argument.
\providecommand*\vfillstretch[1][1]{\vspace*{\stretch{#1}}}
\providecommand*\hfillstretch[1][1]{\hspace*{\stretch{#1}}}

% Abbreviations of latin phrases.
\let\latinabb\empty  % no special formatting
\providecommand*\ie{\latinabb{i.e.}}
\providecommand*\eg{\latinabb{e.g.}}
\providecommand*\etc{\latinabb{etc}}
\providecommand*\vs{\latinabb{vs}}

% General fonts and characters.
\let\longemph\textsl  % alternate way to emphasize
\let\strong\textbf  % strong emphasis
\let\code\texttt  % code fragments
\let\var\textsl  % multi-letter variables
\let\pred\mathbf  % predicates
\providecommand*\stup{\textsuperscript{st}}
\providecommand*\ndup{\textsuperscript{nd}}
\providecommand*\rdup{\textsuperscript{rd}}
\providecommand*\thup{\textsuperscript{th}}
\newcommand*\bbmath[1]{\ensuremath{\mathbb{#1}}}  % requires amssymb
\providecommand*\N{\bbmath{N}}  % requires amssymb
\providecommand*\Z{\bbmath{Z}}  % requires amssymb
\providecommand*\Q{\bbmath{Q}}  % requires amssymb
\providecommand*\R{\bbmath{R}}  % requires amssymb
\providecommand*\bigOh{\mathcal{O}}
\providecommand*\onehalf{\ensuremath{{}^1\!\!/\!_2}}
\providecommand*\letlinebreak{\penalty\exhyphenpenalty}
\providecommand*\halfthinspace{\kern.083333em}
\providecommand*\dash
   {\letlinebreak\halfthinspace---\halfthinspace\letlinebreak}
\let\per\slash  % for convenience

% For defined terms -- requires package ulem.
\newcommand*\defn[1]{\uline{\textit{#1}}}
%\setlength{\ULdepth}{.2ex}

% General math macros.
\newcommand*\hiderel[1]{\mathrel{\hphantom{#1}}}
\providecommand*\comp[1]{\overline{#1}}  % set complement
\providecommand*\emptystr{\varepsilon}  % empty string
\providecommand*\lxor{\mathbin\oplus}  % exclusive or
\providecommand*\cat{\cdot}  % string concatenation
% Floors and ceilings, with optional size-changing command, e.g.,
% \floor[\Big]{x/2} becomes '\Big\lfloor{x/2}\Big\rfloor'.
\providecommand*\floor[2][]{{#1\lfloor}{#2}{#1\rfloor}}
\providecommand*\ceil[2][]{{#1\lceil}{#2}{#1\rceil}}
\providecommand*\lcm{\operatorname{lcm}}  % requires amsmath
\providecommand*\size{\operatorname{size}}  % requires amsmath
\providecommand*\len{\operatorname{len}}  % requires amsmath
\providecommand*\divides{\mathrel{|}}

% Redefined symbols from amssymb.
\renewcommand*\emptyset{\varnothing}  % rounder than default
\let\bigiff\iff
\renewcommand*\iff{\mathrel{\Leftrightarrow}}  % smaller than default
\let\bigimplies\implies
\renewcommand*\implies{\mathrel{\Rightarrow}}  % smaller than default
\renewcommand*\ge{\geqslant}  % with slanted line under the > sign
\renewcommand*\le{\leqslant}  % with slanted line under the < sign

% For algorithms, using either CLRS-style or Python-style pseudocode.
\let\ADT\textsc  % ADT names
\let\proc\textsc  % function names
\let\const\textsc  % constants (True, False, etc.)
\let\kw\textbf  % keywords (if, while, etc.)
\providecommand*\comm[1]{\textsl{\#\space#1}}  % comments
\providecommand*\opgets[1]{\mathrel{#1=}}
\providecommand*\eq{\mathrel{==}}
\providecommand*\True{\const{True}}
\providecommand*\False{\const{False}}
\providecommand*\None{\const{None}}
\providecommand*\cmod{\mathbin{\%}}
\providecommand*\nil{\const{nil}}

% Checkboxes and checklists.
% \checkmark already defined in amssymb
\makeatletter
\@ifpackageloaded{kpfonts}
   {\newcommand*\exmark{$\times$}}
   {\newcommand*\exmark{{\boldmath$\times$}}}
\makeatother
\newlength\exmarksize
\newlength\exmarkdepth
\newcommand*\checkbox[1][]
   {\settoheight\exmarksize{\exmark}
    \settodepth\exmarkdepth{\exmark}
    \addtolength\exmarksize{-\exmarkdepth}
    {\setlength\fboxrule{.1ex}\setlength\fboxsep{.1ex}%
    \fbox{\rule{0pt}{\exmarksize}\makebox[\exmarksize]{\smash{#1}}}}}
\newcommand*\checkedbox{\checkbox[\raisebox{.1ex}{\kern.2em$\checkmark$}]}
\newcommand*\exedbox{\checkbox[\exmark]}
\newenvironment*{checklist}{\begin{list}{\checkbox}{}}{\end{list}}
\newcommand*\checkeditem{\item[\checkedbox]}
\newcommand*\exeditem{\item[\exedbox]}
\newcommand*\boxeditem[1][]{\item[{\checkbox[#1]}]}

% Macros for drawing "underline" rules and half-boxes.
\providecommand*\urule[2][.5pt]{\rule[-.4ex]{#2}{#1}} % "underline" rule
% Underlined text box (with optional positioning argument).
\providecommand*\ubox[3][c]{\rlap{\urule{#2}}\makebox[#2][#1]{#3}}
% Underlined text box with caption (and optional caption positioning).
\providecommand*\capbox[5][c]{\ifx\empty#2\empty
    \else\rlap{\raisebox{-\baselineskip}{\makebox[#3][#1]{#2}}}\fi
    \ubox[#4]{#3}{#5}}

% For student numbers.
\providecommand*\hb{\urule{1.2em}}      % horizontal bar
\providecommand*\vb{\urule[1ex]{.5pt}}  % vertical bar
\providecommand*\studentnumberboxes     % now 10-digits long!
   {\vb\hb\vb\hb\vb\hb\vb\hb\vb\hb\vb\hb\vb\hb\vb\hb\vb\hb\vb\hb\vb}
\newlength\numberboxwidth
\settowidth\numberboxwidth{\studentnumberboxes}

% Command to format sample solutions for tutorials.
\newcommand\samplesolution[1]
   {\ifsolutions\begin{quote}\sffamily{#1}\end{quote}\fi}

% Marking-related macros.
\providecommand*\markitem[2][]
   {\item{\bfseries#2}\ifx\empty#1\empty\else
    \space[$#1$ mark\ifnum#1>1 s\fi]\fi:\quad\ignorespaces}
\providecommand*\errorcode[2][]
   {\item{\bfseries error code #2}\ifx\empty#1\empty\else
    \space[#1]\fi:\quad\ignorespaces}
\providecommand*\commonerror[1][]
   {\item{\bfseries common error}\ifx\empty#1\empty\else
    \space[#1]\fi:\quad\ignorespaces}
\makeatletter
\providecommand*\heading[1]%
   {\@startsection{heading}{9}{0pt}{-.75ex plus -1.5ex minus -.5ex}
    {-.5em plus -.5em minus -.25em}{\normalfont}*{\textsc{#1}}\mbox{}}
\makeatother
