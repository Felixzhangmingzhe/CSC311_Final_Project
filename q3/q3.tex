\documentclass{article}
\usepackage{fullpage,amsmath,amssymb}
\usepackage{hyperref}
\usepackage[none]{hyphenat}
\usepackage{calc}  % arithmetic in length parameters
\usepackage{caption}
\usepackage{enumitem}  % more control over list formatting
\usepackage{fancyhdr}  % simpler headers and footers
\usepackage{geometry}  % page layout
\usepackage{lastpage}  % for last page number
\usepackage{listings}
\usepackage{parskip}
\usepackage{relsize}  % easier font size changes
\usepackage[normalem]{ulem}  % smarter underlining
\usepackage{url}  % verb-like typesetting of URLs
\usepackage{xcolor}
\usepackage{xfrac}  % nicer looking simple fractions for text and math

\definecolor{codegray}{rgb}{0.5,0.5,0.5}
\definecolor{codegreen}{rgb}{0,0.6,0}
\definecolor{codepurple}{rgb}{0.58,0,0.82}
\definecolor{backcolour}{rgb}{0.95,0.95,0.92}

\lstdefinestyle{mystyle}{
    backgroundcolor=\color{backcolour},   
    commentstyle=\color{codegreen},
    keywordstyle=\color{magenta},
    numberstyle=\tiny\color{codegray},
    stringstyle=\color{codepurple},
    basicstyle=\ttfamily\footnotesize,
    breakatwhitespace=false,         
    breaklines=true,                 
    captionpos=b,                    
    keepspaces=false,                 
    numbers=left,                    
    numbersep=5pt,                  
    showspaces=false,                
    showstringspaces=false,
    showtabs=false,                  
    tabsize=4
}
\lstset{style=mystyle}

\everymath{\displaystyle}

\newcommand{\arrayex}[1]{
    \begin{tabular}{|*{20}{c|}}
    \hline
    #1 \\
    \hline
    \end{tabular}
}

\setlength{\tabcolsep}{5pt}

\renewcommand{\arraystretch}{1}

%\usepackage[T1]{fontenc}  % use true 8-bit fonts
%\usepackage{slantsc}  % allow slanted small-caps
%\usepackage{microtype}  % perform various font optimizations
%% Use Palatino-based monospace instead of kpfonts' default.
%\usepackage{newpxtext}

% Common macros.
\input{macros-263}


\geometry{a4paper, margin=1in, headheight=15pt, headsep=20pt}

\pagestyle{fancy}
\fancyhf{}
\fancyhead[L]{CSC311 Summer 2024}
\fancyhead[R]{Final Project}
\fancyfoot[C]{\thepage}


\title{Question 3}
\date{\vspace{-10.0ex}}
\hypersetup{pdfpagemode=Fullscreen,
    colorlinks=true,
    linkfileprefix={}}


\begin{document}
\maketitle
\thispagestyle{fancy}


\textbf{We choose Option 2}
\begin{enumerate}[label=(\alph*)]
    \item 
    \begin{itemize}
        \item ALS breaks down large matrices into lower-dimensional matrices, while neural networks model non-linear relationships through layers.
        
        \item ALS is less flexible than neural networks because it is designed for matrix factorization, whereas neural networks can model non-linear relationships.
        
        \item ALS is more computationally efficient than neural networks for sparse datasets because neural networks require significant computational resources.
    \end{itemize}

    \item The implementation is in \texttt{neural\_network.py}.
    
    \item The optimization hyperparameters we chose are:
    
    \texttt{k = 50, lr = 0.01, num\_epoch = 50}

    The Validation Accuracy we obtained is 0.68981.

    \item The plot with \texttt{k = 50, lr = 0.01, num\_epoch = 50} is shown below:
    
    \includegraphics[width=0.7\linewidth]{6031723149317_.pic.jpg}

    \includegraphics[width=0.7\linewidth]{6041723149328_.pic.jpg}

    The Final Test Accuracy is 0.68558.

    \item The best regularization penalty we found is $\lambda = 0.01$. With this value of $\lambda$, we obtained:
    
    Final Validation Accuracy: 0.67824

    Final Test Accuracy: 0.68078

    The model performed about the same with the regularization penalty. This may be because our model is already well-regularized and does not overfit, or only has negligible overfitting issues.
\end{enumerate}

\end{document}
