\documentclass{article}
\usepackage{fullpage,amsmath,amssymb}
\usepackage{hyperref}
\usepackage[none]{hyphenat}
\usepackage{calc}  % arithmetic in length parameters
\usepackage{caption}
\usepackage{enumitem}  % more control over list formatting
\usepackage{fancyhdr}  % simpler headers and footers
\usepackage{geometry}  % page layout
\usepackage{lastpage}  % for last page number
\usepackage{listings}
\usepackage{relsize}  % easier font size changes
\usepackage[normalem]{ulem}  % smarter underlining
\usepackage{url}  % verb-like typesetting of URLs
\usepackage{xcolor}
\usepackage{xfrac}  % nicer looking simple fractions for text and math

\definecolor{codegray}{rgb}{0.5,0.5,0.5}
\definecolor{codegreen}{rgb}{0,0.6,0}
\definecolor{codepurple}{rgb}{0.58,0,0.82}
\definecolor{backcolour}{rgb}{0.95,0.95,0.92}

\lstdefinestyle{mystyle}{
    backgroundcolor=\color{backcolour},   
    commentstyle=\color{codegreen},
    keywordstyle=\color{magenta},
    numberstyle=\tiny\color{codegray},
    stringstyle=\color{codepurple},
    basicstyle=\ttfamily\footnotesize,
    breakatwhitespace=false,         
    breaklines=true,                 
    captionpos=b,                    
    keepspaces=false,                 
    numbers=left,                    
    numbersep=5pt,                  
    showspaces=false,                
    showstringspaces=false,
    showtabs=false,                  
    tabsize=4
}
\lstset{style=mystyle}

\everymath{\displaystyle}

\newcommand{\arrayex}[1]{
    \begin{tabular}{|*{20}{c|}}
    \hline
    #1 \\
    \hline
    \end{tabular}
}

\setlength{\tabcolsep}{5pt}

\renewcommand{\arraystretch}{1}

%\usepackage[T1]{fontenc}  % use true 8-bit fonts
%\usepackage{slantsc}  % allow slanted small-caps
%\usepackage{microtype}  % perform various font optimizations
%% Use Palatino-based monospace instead of kpfonts' default.
%\usepackage{newpxtext}

% Common macros.
\input{macros-263}


\geometry{a4paper, margin=1in, headheight=15pt, headsep=20pt}

\pagestyle{fancy}
\fancyhf{}
\fancyhead[L]{CSC311 Summer 2024}
\fancyhead[R]{Final Project}
\fancyfoot[C]{\thepage}


\title{Question 1}
\date{\vspace{-7.5ex}}
\hypersetup{pdfpagemode=Fullscreen,
    colorlinks=true,
    linkfileprefix={}}


\begin{document}
\maketitle
\thispagestyle{fancy}

\begin{enumerate}
    \item [(a) (b) (c)] The accuracy on the validation data with $k \in \{1, 6, 11, 16, 21, 26\}$ on user-based and item-based collaborative filtering is as follows:
    \begin{center}
        \includegraphics[width=0.8\textwidth]{q1.png}
    \end{center}
    Test Accuracy on user-based CF with k* = 11: 0.6841659610499576

    Test Accuracy on item-based CF with k* = 21: 0.6816257408975445

    \item [(d)] The test on user-based CF is slightly better than item-based CF.
    
    Additionally, the test accuracy on user-based CF cost less time than item-based CF.

    Therefore, user-based CF is better than item-based CF in this case.

    \item [(e)] 
    \begin{itemize}
        \item The \textsc{knn} algorithm is computational expensive for large datasets.
        \item The Curse of Dimensionality: In high dimensions, ``most'' points are approximately the same distance and the nearest neighbors are not very useful.
    \end{itemize}
\end{enumerate}

\end{document}
