\documentclass{article}
\usepackage{fullpage,amsmath,amssymb}
\usepackage{hyperref}
\usepackage[none]{hyphenat}
\usepackage{calc}  % arithmetic in length parameters
\usepackage{caption}
\usepackage{enumitem}  % more control over list formatting
\usepackage{fancyhdr}  % simpler headers and footers
\usepackage{geometry}  % page layout
\usepackage{lastpage}  % for last page number
\usepackage{listings}
\usepackage{relsize}  % easier font size changes
\usepackage[normalem]{ulem}  % smarter underlining
\usepackage{url}  % verb-like typesetting of URLs
\usepackage{xcolor}
\usepackage{xfrac}  % nicer looking simple fractions for text and math

\definecolor{codegray}{rgb}{0.5,0.5,0.5}
\definecolor{codegreen}{rgb}{0,0.6,0}
\definecolor{codepurple}{rgb}{0.58,0,0.82}
\definecolor{backcolour}{rgb}{0.95,0.95,0.92}

\lstdefinestyle{mystyle}{
    backgroundcolor=\color{backcolour},   
    commentstyle=\color{codegreen},
    keywordstyle=\color{magenta},
    numberstyle=\tiny\color{codegray},
    stringstyle=\color{codepurple},
    basicstyle=\ttfamily\footnotesize,
    breakatwhitespace=false,         
    breaklines=true,                 
    captionpos=b,                    
    keepspaces=false,                 
    numbers=left,                    
    numbersep=5pt,                  
    showspaces=false,                
    showstringspaces=false,
    showtabs=false,                  
    tabsize=4
}
\lstset{style=mystyle}

\everymath{\displaystyle}

\newcommand{\arrayex}[1]{
    \begin{tabular}{|*{20}{c|}}
    \hline
    #1 \\
    \hline
    \end{tabular}
}

\setlength{\tabcolsep}{5pt}

\renewcommand{\arraystretch}{1}

%\usepackage[T1]{fontenc}  % use true 8-bit fonts
%\usepackage{slantsc}  % allow slanted small-caps
%\usepackage{microtype}  % perform various font optimizations
%% Use Palatino-based monospace instead of kpfonts' default.
%\usepackage{newpxtext}

% Common macros.
\input{macros-263}


\geometry{a4paper, margin=1in, headheight=15pt, headsep=20pt}

\pagestyle{fancy}
\fancyhf{}
\fancyhead[L]{CSC311 Summer 2024}
\fancyhead[R]{Final Project}
\fancyfoot[C]{\thepage}


\title{Question 2}
\date{\vspace{-7.5ex}}
\hypersetup{pdfpagemode=Fullscreen,
    colorlinks=true,
    linkfileprefix={}}


\begin{document}
\maketitle
\thispagestyle{fancy}

\begin{enumerate}[label=(\alph*)]
    \item Given the probability that the question $j$ is correctly answered by student $i$ is:
    \begin{equation*}
        p_{ij} =  \frac{\exp(\theta_i - \beta_j)}{1 + \exp(\theta_i - \beta_j)}
    \end{equation*}
    
    The log-likelihood for all students is derived as follows:
    \begin{align*}
        \log p(\mathbf{C}|\boldsymbol{\theta}, \boldsymbol{\beta}) &= \sum_{i, j} (c_{ij} \log p_{ij} + (1 - c_{ij}) \log (1 - p_{ij})) \\
        &= \sum_{i=1}^{n} \sum_{j=1}^{m} \left( c_{ij} \log \left( \frac{\exp(\theta_i - \beta_j)}{1 + \exp(\theta_i - \beta_j)} \right) + (1 - c_{ij}) \log \left( 1 - \frac{\exp(\theta_i - \beta_j)}{1 + \exp(\theta_i - \beta_j)} \right) \right) \\
        &= \sum_{i=1}^{n} \sum_{j=1}^{m} (c_{ij} (\theta_i - \beta_j) - \log(1 + \exp(\theta_i - \beta_j))),
    \end{align*}
    where $c_{ij}$ is the binary response of student $i$ to question $j$.

    The log-likelihood with respect to $\theta_i$ is:
    \begin{align*}
        \frac{\partial \log p(\mathbf{C}|\boldsymbol{\theta}, \boldsymbol{\beta})}{\partial \theta_i} &= \sum_{j=1}^{m} \left( c_{ij} - \frac{\exp(\theta_i - \beta_j)}{1 + \exp(\theta_i - \beta_j)} \right) \\
        &= \sum_{j=1}^{m} (c_{ij} - p_{ij}).
    \end{align*}

    The log-likelihood with respect to $\beta_j$ is:
    \begin{align*}
        \frac{\partial \log p(\mathbf{C}|\boldsymbol{\theta}, \boldsymbol{\beta})}{\partial \beta_j} &= \sum_{i=1}^{n} \left( c_{ij} - \frac{\exp(\theta_i - \beta_j)}{1 + \exp(\theta_i - \beta_j)} \right) \\
        &= \sum_{i=1}^{n} (c_{ij} - p_{ij}).
    \end{align*}

    \item The hyperparameters I selected are: learning rate = 0.01 and iterations = 100.
    
    The training and validation accuracies vs iterations are in the graph below:
    \begin{center}
        \includegraphics[width=0.6\textwidth]{irt_accuracy.png}
    \end{center}

    The log-likelihoods vs iterations are in the graph below:
    \begin{center}
        \includegraphics[width=0.6\textwidth]{irt_llk.png}
    \end{center}

    \item The Final Validation Accuracy: 0.7063223257126728
    
    The Final Test Accuracy: 0.707310189105278

    \item I select the lowest difficulty question $j_1$ (Question 1165), the highest difficulty question $j_2$ (Question 47852) and the average difficulty question $j_3$ (Question 1410).

    The probability of the correct response is in the graph below:
    \begin{center}
        \includegraphics[width=0.6\textwidth]{irt_question.png}
    \end{center}

    \item The shape of the curves are like the sigmoid function as expected.
    
    Fix a question $j$. As $\theta_i$ increases, the probability of the correct response $p_{ij}$ increases. This means if a student has a higher ability, the probability of the correct response increases.

    Fix a student $i$. As $\beta_j$ increases, the probability of the correct response $p_{ij}$ decreases. This means if a question has a higher difficulty, the probability of the correct response decreases.
\end{enumerate}

\end{document}
